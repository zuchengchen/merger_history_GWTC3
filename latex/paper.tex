\documentclass[twocolumn]{aastex631}
%\documentclass[twocolumn,linenumbers]{aastex631}
%%%%%%%%%%%%%%%%%%%%%%%%%%%%%%% load packages %%%%%%%%%%%%%%%%%%%%%%%%%%%%%%%%%%
%\usepackage{ulem}
%\hypersetup{linkcolor=red,citecolor=blue,filecolor=cyan,urlcolor=magenta}
\usepackage{amsmath}
%\usepackage{subfigure}
%\usepackage{float}
%\usepackage{tensor}     % manipulate tensors
%\usepackage{graphicx}   % include figures
%\usepackage{bm}         % \bm{<text>} Bold math symbols
%\usepackage{xcolor}     % textcolor
%\usepackage{color}      % \textcolor{declared-color}{text}
%\usepackage{longtable}  % large table\
%\usepackage{fontspec} % accented characters for .bib file using XeLaTex
%\usepackage[section]{placeins} % figures processing
%\usepackage[utf8]{inputenc} % bibtex formatting

%%%%%%%%%%%%%%%%%%%%%%%%%%%%%%%%%%%%%%%%%%%%%%%%%%%%%%%%%%%%%%%%%%%%%%%%%%%%%%%%
%%%%%%%%%%%%%%%%%%%%%%%%%%%%%%%%% new commands %%%%%%%%%%%%%%%%%%%%%%%%%%%%%%%%%
\newcommand{\nc}{\newcommand*} 

%%%%%%%%%%%%%%%%%%%%%%%%%%%%%%%%%% alphabet %%%%%%%%%%%%%%%%%%%%%%%%%%%%%%%%%%%
\nc{\al}{\alpha}
\nc{\s}{\sigma}
\nc{\dt}{\delta}
\nc{\Dt}{\Delta}
\nc{\Ld}{\Lambda}
\nc{\p}{\partial}
\nc{\om}{\omega}
\nc{\Om}{\Omega}
\nc{\rd}{\mathrm{d}}
\nc{\Od}[1]{\mathcal{O}(#1)} % order operator
\nc{\kp}{\kappa}

%%%%%%%%%%%%%%%%%%%%%%%%%%%%%%%%%% equations %%%%%%%%%%%%%%%%%%%%%%%%%%%%%%%%%%
\def\({\left(}
\def\){\right)}
\def\[{\left[}
\def\]{\right]}
\def\e{\begin{equation}}
\def\q{\end{equation}}
\def\m{\begin{eqnarray}}
\def\n{\end{eqnarray}}
\def\permille{\ensuremath{{}^\text{o}\mkern-5mu/\mkern-3mu_\text{oo}}}

%%%%%%%%%%%%%%%%%%%%%%%%%%%%%%%%%% citations %%%%%%%%%%%%%%%%%%%%%%%%%%%%%%%%%%%
\nc{\Eq}[1]{Eq.~\eqref{#1}}     % equation
\nc{\Fig}[1]{Fig.~\ref{#1}}     % figure
\nc{\Table}[1]{Table~\ref{#1}}  % table
\nc{\Sec}[1]{Sec.~\ref{#1}}     % section

%%%%%%%%%%%%%%%%%%%%%%%%%%%%%%% gravitational waves %%%%%%%%%%%%%%%%%%%%%%%%%%%
\nc{\Msun}{M_\odot}             % solar mass
\nc{\fpbh}{f_{\mathrm{pbh}}}    % f_pbh
\nc{\fpbhn}{f_{\mathrm{pbh0}}}    % f_pbh
\nc{\mR}{\mathcal{R}} % merger rate density
\nc{\seq}{\sigma_{\mathrm{eq}}}
\nc{\ogw}{\Omega_{\mathrm{GW}}}
\nc{\gpcyr}{\mathrm{Gpc}^{-3}\,\mathrm{yr}^{-1}}
\nc{\lvc}{LIGO/Virgo} % LIGO-VIRGO collaboration
\nc{\SNR}{\mathrm{SNR}} % signal to noise ratio
\nc{\mmin}{{m_{\mathrm{min}}}}
\nc{\mmax}{{m_{\mathrm{max}}}}
\nc{\Mmin}{{M_{\mathrm{min}}}}
\nc{\fmin}{{f_{\mathrm{min}}}}
\nc{\VT}{\mathrm{VT}}
\nc{\rhoGW}{\rho_{\mathrm{GW}}}
\nc{\vth}{\vec{\theta}}
\nc{\vd}{\vec{d}}
\nc{\vla}{\vec{\lambda}}
\nc{\Nobs}{N_{\mathrm{obs}}}
\nc{\av}[1]{\langle #1 \rangle} % average bracket
\nc{\km}{\mathrm{km}}
\nc{\Mpc}{\mathrm{Mpc}}
\nc{\Tobs}{T_{\mathrm{obs}}}
\nc{\Ntemp}{N_{\mathrm{temp}}}

%%%%%%%%%%%%%%%%%%%%%%%%%%%%%%%%%%%% other %%%%%%%%%%%%%%%%%%%%%%%%%%%%%%%%%%%%
\nc{\addref}{[\textcolor{red}{add ref}] } % placeholder of references
\nc{\eg}{\textit{e.g.~}}
\nc{\app}{\approx}
\nc{\hf}{\frac{1}{2}}
\nc{\discuss}{\textcolor{red}{Add discussion here!}}
\nc{\red}[1]{\textcolor{red}{#1}}

%%%%%%%%%%%%%%%%%%%%%%%%%%% only used in this paper %%%%%%%%%%%%%%%%%%%%%%%%%%%%
\nc{\mH}{\mathcal{H}}
\nc{\cs}{c_s^2}
\nc{\Sij}[1]{S_{ij}^{(#1)}}
\nc{\vi}[1]{v_i^{(#1)}}
\nc{\no}{\nonumber}
\def\<{\left\langle}
\def\>{\right\rangle}
\def\ap{\alpha}
\def\half{{1\over 2}}
\nc{\bk}{\bm{k}}
\nc{\bq}{\bm{q}}
\nc{\bp}{\bm{p}}
\nc{\bl}{\bm{l}}
\nc{\bx}{\bm{x}}
\nc{\be}{\mathbf{e}}
\nc{\mS}{\mathcal{S}}
\nc{\te}{\tilde{\eta}}
\nc{\tp}{\tilde{p}}
\nc{\tk}{\tilde{k}}
\nc{\tx}{\tilde{x}}
\nc{\tF}{\tilde{F}}
\nc{\tA}{\tilde{A}}
\nc{\mkpq}{|\bk-\bp-\bq|}
\nc{\mpq}{|\bp-\bq|}
\nc{\mkp}{|\bk-\bp|}
\nc{\mSi}[1]{\mS^{(#1)}({\bk, \eta})}
\nc{\vk}{\vec{k}}
\nc{\kstar}{k_*}
\nc{\xstar}{x_*}
\nc{\mpbh}{m_{\rm{pbh}}}
\nc{\cR}{\mathcal{R}}
\nc{\mU}{{\mathcal{U}}}
\nc{\Mc}{{M_\mathrm{c}}}
\nc{\Mf}{{M_\mathrm{f}}}
%%%%%%%%%%%%%%%%%%%%%%%%%%%%%%%%%%%%%%%%%%%%%%%%%%%%%%%%%%%%%%%%%%%%%%%%%%%%%%%%
%%%%%%%%%%%%%%%%%%%%%%%%%%%%%%%%%% main text %%%%%%%%%%%%%%%%%%%%%%%%%%%%%%%%%%%

%%%%%%%%%%%%%%%%%%%%%%%%%%%%%%%%%%%%%%%%%%%%%%%%%%%%%%%%%%%%%%%%%%%%%%%%%%%%%%%%
%% Reintroduced the \received and \accepted commands from AASTeX v5.2
%\received{}
%\revised{}
%\accepted{}
%\published{}
%% Command to document which AAS Journal the manuscript was submitted to.
%% Adds "Submitted to " the arguement.
%\submitjournal{ApJ}

%%%%%%%%%%%%%%%%%%%%%%%%%%%%%%%%%%%%%%%%%%%%%%%%%%%%%%%%%%%%%%%%%%%%%%%%%%%%%%%%
\begin{document}
%%%%%%%%%%%%%%%%%%%%%%%%%%%%%%%%%%%%%%%%%%%%%%%%%%%%%%%%%%%%%%%%%%%%%%%%%%%%%%%%
	
%%%%%%%%%%%%%%%%%%%%%%%%%%%%%%%%%%%% title %%%%%%%%%%%%%%%%%%%%%%%%%%%%%%%%%%%%%	
\title{Constrain the Merger History of Primordial-Black-Hole Binaries from GWTC-3}
	
%%%%%%%%%%%%%%%%%%%%%%%%%%%%%%%%%%%% author %%%%%%%%%%%%%%%%%%%%%%%%%%%%%%%%%%%%
\author[0000-0002-0297-9633]{Lang~Liu}
\email{liulang@bnu.edu.cn}
\affiliation{Department of Astronomy, Beijing Normal University, Beijing 100875, China}
\affiliation{Advanced Institute of Natural Sciences, Beijing Normal University, Zhuhai 519087, China}


%%%%%%%%%%%%%%%%%%%%%%%%%%%%%%%%%%%% author %%%%%%%%%%%%%%%%%%%%%%%%%%%%%%%%%%%%
\author[0000-0002-3309-415X]{Zhi-Qiang~You}
\email{zhiqiang.you@bnu.edu.cn}
\affiliation{Department of Astronomy, Beijing Normal University, Beijing 100875, China}
\affiliation{Advanced Institute of Natural Sciences, Beijing Normal University, Zhuhai 519087, China}

%%%%%%%%%%%%%%%%%%%%%%%%%%%%%%%%%%%% author %%%%%%%%%%%%%%%%%%%%%%%%%%%%%%%%%%%%
\author[0000-0002-9610-2284]{You~Wu}
\email{youwuphy@gmail.com}
\affiliation{College of Mathematics and Physics, Hunan University of Arts and Science, Changde, 415000, China}

%%%%%%%%%%%%%%%%%%%%%%%%%%%%%%%%%%%% author %%%%%%%%%%%%%%%%%%%%%%%%%%%%%%%%%%%%
\author[0000-0001-7016-9934]{Zu-Cheng~Chen}
\email{Corresponding author: zucheng.chen@bnu.edu.cn}
\affiliation{Department of Astronomy, Beijing Normal University, Beijing 100875, China}
\affiliation{Advanced Institute of Natural Sciences, Beijing Normal University, Zhuhai 519087, China}

	
%\date{\today}
%%%%%%%%%%%%%%%%%%%%%%%%%%%%%%%%%%%%%%%%%%%%%%%%%%%%%%%%%%%%%%%%%%%%%%%%%%%%%%%%
\begin{abstract}
Primordial black holes (PBHs), as dark matter candidates, have attracted 
more and more attentions as they could be possible progenitors of 
the heavy binary black holes (BBHs) observed by \lvc.
Accurately estimating the merger rate of PBH binaries will be crucial to
reconstruct the mass distribution of PBHs.
%However there exists some uncertainties in estimating the merger rate.
It was pointed out the merger history of PBHs may shift the merger rate distribution depending on the mass function of PBHs.
In this paper, we use BBH events from \lvc\ GWTC-3 data release to
constrain the merger rate distribution of PBHs by taking into account the effect of merger history.
It is found that the second merger process makes subdominant contribution to the 
total merger rate, and hence the merger history effect can be safely neglected.	
We also confirm that the main components of dark matter should not be made of stellar 
mass PBHs.

%\keywords{primordial black holes, merger rate, merger history}
\end{abstract}
	
%\maketitle

%%%%%%%%%%%%%%%%%%%%%%%%%%%%%%%%%%%%%%%%%%%%%%%%%%%%%%%%%%%%%%%%%%%%%%%%%%%%%%%%
%%%%%%%%%%%%%%%%%%%%%%%%%%%%%%%%%%%%%%%%%%%%%%%%%%%%%%%%%%%%%%%%%%%%%%%%%%%%%%%%
\section{Introduction}
The direct detection of gravitational wave (GW) from a binary black hole (BBH)
coalescence \cite{Abbott:2016blz} has opened a new window into the universe.
Over the past few years, ten BBH mergers have been reported by \lvc\ during 
the O1 and O2 observing runs \cite{Abbott:2016blz,Abbott:2016nmj,%
TheLIGOScientific:2016pea,Abbott:2017vtc,%
Abbott:2017gyy,Abbott:2017oio,LIGOScientific:2018mvr}.
The progenitors of these BBHs are still unknown and under intensively investigation (see \eg \cite{Bird:2016dcv,Sasaki:2016jop,Chen:2018czv,%
Fishbach:2017dwv,Clesse:2016vqa,Antonini:2016gqe,%
Inayoshi:2017mrs,Ali-Haimoud:2017rtz,Perna:2019axr,Kavanagh:2018ggo,%
Rodriguez:2015oxa,Rodriguez:2016kxx,Park:2017zgj,%
Belczynski:2014iua,Belczynski:2016obo,Woosley:2016nnw,%
Rodriguez:2018rmd,Choksi:2018jnq,2010AIPC.1314..291D,deMink:2016vkw}).
These \lvc\ BBHs present a much heavier mass distribution (in particular the 
source-frame primary mass of GW170729 event can be as heavy as
$50.2^{+16.2}_{-10.2} \Msun$ \cite{LIGOScientific:2018mvr}) than that 
inferred from X-ray observations
\cite{Wiktorowicz:2013dua,Casares:2013tpa,Corral-Santana:2013uua,%
Corral-Santana:2015fud},
which would challenge the formation and evolution mechanisms of astrophysical
black holes. 
One possible explanation for \lvc\ BBHs is the primordial black holes (PBHs)
\cite{Bird:2016dcv,Sasaki:2016jop,Chen:2018czv}
formed through the gravitational collapse of the primordial density
fluctuations \cite{Hawking:1971ei,Carr:1974nx}, 
which may accompany the induced GWs
\cite{Yuan:2019udt,Yuan:2019wwo,Chen:2019xse,Yuan:2019fwv,Cai:2019bmk,Cai:2018dig,Cai:2019elf,DeLuca:2019ufz,Bartolo:2018rku,Bartolo:2018evs}.
On the other hand, PBHs can also be a candidate of cold dark matter (CDM),
and the abundance of PBHs in CDM has been constrained by various experiments
\cite{Carr:2009jm,Barnacka:2012bm,Graham:2015apa,Niikura:2017zjd,%
Griest:2013esa,Niikura:2019kqi,Tisserand:2006zx,Brandt:2016aco,Gaggero:2016dpq,%
Ali-Haimoud:2016mbv,Blum:2016cjs,Horowitz:2016lib,Chen:2016pud,%
Wang:2016ana,Abbott:2018oah,Magee:2018opb,Wang:2019kaf,Chen:2018rzo,Chen:2019irf,%
Yuan:2019udt,Chen:2019xse,Montero-Camacho:2019jte,Laha:2019ssq,Dasgupta:2019cae,Gow:2019pok}.

In order to account for the \lvc\ BBHs, the merger rate of PBH binaries has
been estimated to be $17 \sim 288~\gpcyr$ \cite{Chen:2018rzo}.
One should notice that theoretically there exist some uncertainties in
estimating the merger rate distribution of PBH binaries, and the accuracy of 
the estimation has been continuously improved.
The merger rate of PBH binaries with monochromatic mass function has already
been given in \cite{Nakamura:1997sm,Sasaki:2016jop} for the case where two 
neighboring PBHs having sufficiently small separation can form a binary in
the early Universe due to the torque from the third nearest PBH.
These binaries would then evolve and coalesce within the age of the Universe 
and finally explain the merger events observed by \lvc\
\cite{Sasaki:2016jop}.
Later, the merger rate estimation is improved in \cite{Ali-Haimoud:2017rtz} 
by taking into account the torques exerted by all CDM (including all the PBHs
and linear density perturbations), but it is also assumed that all PBHs have
the same mass.
It is also pointed out in \cite{Ali-Haimoud:2017rtz} that the effects such as
encountering with other PBHs, tidal field from the smooth halo, and
the baryon accretion are subdominant and can be neglected when estimating
the merger rate.

Various attempts have been made to estimate the merger rate distribution
of PBH binaries when PBHs have an extended mass function 
\cite{Chen:2018czv,Chen:2018rzo,Raidal:2018bbj,Raidal:2017mfl,Liu:2018ess,Liu:2020cds,Vaskonen:2019jpv}.
In particular, a formalism to estimate the effect of merger history of PBHs
on merger rate distribution has been developed in \cite{Liu:2019rnx},
and it is argued that the multiple-merger effect may not be ignored if
PBHs have a power-law or a log-normal mass function by choosing some specific
parameter values of the mass function.
An accurate estimation of the merger rate will be crucial to infer the event
rate of \lvc\ BBHs and constrain the abundance of PBHs in CDM either through
null detection of sub-solar mass PBH binaries or the null detection
of stochastic GW background (SGWB) from PBH binaries.

In this paper, we will use the LVK recent released GWTC-3 data to constrain the effect of merger history on the merger rate of PBH binaries assuming all of LVK BBHs are of primordial origin.
We organize the rest paper as follows.
In \Sec{merger}, we give a brief review on calculation of merger rate of PBH binaries by accounting for the merger history effect.
In \Sec{method}, we describe the hierarchical Bayesian framework used to infer the PBH population parameters from GW data.
In \Sec{result}, we consider four commonly used PBH mass functions and present the results.
Finally, we give conclusions in \Sec{conclusion}.

%%%%%%%%%%%%%%%%%%%%%%%%%%%%%%%%%%%%%%%%%%%%%%%%%%%%%%%%%%%%%%%%%%%%%%%%%%%%%%%%
\section{\label{merger}Merger rate density distribution of PBH binaries}

In this section, we will outline the calculation of merger rate density when taking into account the PBH merger history effect. We refer to \cite{Liu:2019rnx} for more details.

The BBHs observed by LVK suggest that BHs should have a broad mass distribution, so we consider an extended mass function for PBHs. Here, we demand the probability distribution function of PBH mass, $P(m)$, be normalized such that
\e\label{norm}
    \int_{0}^{\infty} P(m)\, \rd m = 1.
\q 
Assuming the fraction of PBHs in CDM is $\fpbh$, we can estimate the abundance of PBHs in the mass interval $(m, m+\rd m)$ as \citep{Chen:2018rzo}
\e 
    0.85 \fpbh\, P(m)\, \rd m,
\q 
where the coefficient $0.85$ is roughly the fraction of CDM in non-relativistic matter including both CDM and baryons. Following \cite{Liu:2019rnx}, we may define an average PBH mass, $\mpbh$, as
\e\label{mpbh}
    \frac{1}{\mpbh} = \int \frac{P(m)}{m} \rd m.
\q 
Then, we can obtain the average number density of PBHs with mass $m$ in the 
total number density of PBHs, $F(m)$, by \citep{Liu:2019rnx}
\e\label{Fm} 
    F(m) = P(m) \frac{\mpbh}{m}.
\q 

We can now proceed to estimate the merger rate densities of PBH binaries by considering the merger history effect.
We assume that PBHs are randomly distributed in the early Universe when they decouple from the cosmic background evolution \citep{Nakamura:1997sm,Sasaki:2016jop,Ali-Haimoud:2017rtz}. Two nearest PBHs would attract each other because of the gravitational interactions. These two PBHs would obtain the angular momentum from the torque of other PBHs and form a PBH binary. The binary would emit gravitational radiations and eventually merge. 

We do not intend to give a detailed derivation, but quote the results from \cite{Liu:2019rnx} here.
The merger rate density from first-merger process, $\cR_1(t, m_i, m_j)$, is given by \citep{Liu:2019rnx}
\e 
\cR_1(t, m_i, m_j) = \int \hat{\cR}_1 \rd m_l,
\q 
where
\e 
\begin{split}
    &\hat{\cR}_1(t, m_i, m_j, m_l)
    \equiv 1.32 \times 10^6 \times \(\frac{t}{t_0}\)^{-\frac{34}{37}}\(\frac{\fpbh}{\mpbh}\)^\frac{53}{37} \\ 
    &\times m_l^{-\frac{21}{37}} (m_i m_j)^\frac{3}{37} (m_i + m_j)^\frac{36}{37} F(m_i) F(m_j) F(m_l).     
\end{split}
\q
Similarly, the merger rate density from second-merger process, $\cR_2(t, m_i, m_j)$, is given by \citep{Liu:2019rnx}
\e 
\begin{split}
    \cR_2(t, m_i, m_j) &= \hf \int \hat{\cR}_2(t, m_i-m_e, m_e, m_j, m_l)\ \rd m_l \rd m_e\\
    &+ \hf \int \hat{\cR}_2(t, m_j-m_e, m_e, m_i, m_l)\ \rd m_l \rd m_e,
\end{split}
\q 
where
\e 
\begin{split}
    &\hat{\cR}_2(t, m_i, m_j,m_k, m_l)
    =1.59 \times 10^4 \times \(\frac{t}{t_0}\)^{-\frac{31}{37}} \(\frac{\fpbh}{\mpbh}\)^\frac{69}{37}\\ 
    &\qquad\times m_k^{\frac{6}{37}} m_l^{-\frac{42}{37}} (m_i+ m_j)^\frac{6}{37} (m_i + m_j + m_k)^\frac{72}{37}\\
    &\qquad\times F(m_i) F(m_j) F(m_k) F(m_l).     
\end{split}
\q
We only consider the effect of merger history up to second-merger process.
Therefore the total merger rate density, $\cR(t, m_i, m_j)$, of PBH binaries at cosmic time $t$ with masses $m_i$ and $m_j$ is
\e\label{cR}
    \cR(t, m_i, m_j) =  \sum_{n=1,2} \cR_n(t, m_i, m_j),
\q 
and the total merger rate is
\e\label{Rt}
    R(t) = \int \cR(t, m_i, m_j) \rd m_i \rd m_j = \sum_{n=1,2} R_n(t),
\q 
where
\e 
 R_n(t) = \int \cR_n(t, m_i, m_j) \rd m_i \rd m_j.
\q 
All the above mentioned merger rate (density) is measured at source frame.
We should emphasize that although $R_2(t)$ should be smaller than $R_1(t)$ as expected, $\cR_2(t, m_i, m_j)$ is not necessarily be smaller than $\cR_1(t, m_i, m_j)$ \citep{Liu:2019rnx}. 

%%%%%%%%%%%%%%%%%%%%%%%%%%%%%%%%%%%%%%%%%%%%%%%%%%%%%%%%%%%%%%%%%%%%%%%%%%%%%%%%
\section{\label{method}Hierarchical Bayesian Inference}
To infer the population parameters, we adopt a hierarchical Bayesian approach by marginalizing over the uncertainty in the estimate of individual event parameters.
In this section, we describe the hierarchical Bayesian inference used in the parameter estimations.
The merger rate density \eqref{cR} is measured in source frame, and we need to convert it into the detector frame as
\e\label{Rpop}
\mR_{\mathrm{pop}}(\theta|\Ld) = \frac{1}{1+z} \frac{dV_\mathrm{c}}{dz} \mR(\theta|\Ld),
\q
where $z$ is the cosmological redshift, $\theta\equiv\{z, m_1, m_2\}$, $\Ld$ is a collection of $\fpbh$ and the parameters from mass function $P(m)$, and $dV_\mathrm{c}/dz$ is the differential comoving volume. The factor $1/(1 + z)$ converts time increments from the source frame to the detector frame. In this work, we take the cosmological parameters from Planck 2018 \citep{Planck:2018vyg}.


Given the data, $\textbf{d} = \{d_1, d_2, \cdots, d_{N_{\mathrm{obs}}}\}$, of $N_{\mathrm{obs}}$ BBH merger events, we model the total number of events as an inhomogeneous Poisson process, yielding the likelihood \citep{Loredo:2004nn,Thrane:2018qnx,Mandel:2018mve}
\begin{equation}\label{L1}
	\mathcal{L}(\textbf{d}|\Ld) \propto N_{\mathrm{exp}}^{N_{\mathrm{obs}}} e^{-N_{\mathrm{exp}}} \prod_{i=1}^{N_{\mathrm{obs}}} \frac{\int \mathcal{L} (d_{i}| \theta)\, \cR_{\mathrm{pop}}(\theta|\Ld) d \theta}{\xi(\Ld)},
\end{equation}
where $N_{\exp}\equiv N_{\exp}(\Ld)$ is the expected number of detections over the timespan of observation, and $\mathcal{L} (d_{i}|\theta)$ is the individual event likelihood for the $i$th GW event that can be derived from the individual event's posterior by reweighing with the prior on $\theta$. Here, $\xi(\Ld)$ quantifies selection biases for a population with parameters $\Ld$ and is defined by
\e\label{xi}
\xi(\Ld) = \int P_{\mathrm{det}}(\theta)\, \cR_{\mathrm{pop}}(\theta|\Ld)\, \mathrm{d} \theta,
\q
where $P_{\text{det}}(\theta)$ is the detection probability that depends on the source parameters $\theta$.
In practice, we use the simulated injections \citep{ligo_scientific_collaboration_and_virgo_2021_5546676} to estimate $\xi(\Ld)$, and \Eq{xi} can be approximated by a Monte Carlo integral over found injections \citep{LIGOScientific:2021psn}
\begin{equation}
	\xi(\Ld) \approx \frac{1}{N_{\mathrm{inj}}} \sum_{j=1}^{N_{\text {found }}} \frac{\cR_{\mathrm{pop}}(\theta_{j} | \Ld)}{p_{\mathrm{draw}}(\theta_j)},
\end{equation}
where $N_{\text{inj}}$ is the total number of injections, $N_{\text{found}}$ is the number of successfully detected injections, and $p_{\mathrm{draw}}$ is the probability density function from which the injections are drawn.
Using the posterior samples from each individual event, the hyper-likelihood \eqref{L1} can be estimated as
\begin{equation}\label{L2}
	\mathcal{L}(\textbf{d}|\Ld) \propto N_{\mathrm{exp}}^{N_{\mathrm{obs}}} e^{-N_{\mathrm{exp}}} \prod_{i=1}^{N_{\mathrm{obs}}} \frac{1}{\xi(\Ld)} \left\langle \frac{\cR_{\mathrm{pop}}(\theta|\Ld)}{d_L^2(z)} \right\rangle,
\end{equation}
where $\langle\cdots\rangle$ denotes the weighted average over posterior samples of $\theta$. The denominator $d_L^2(z)$ is the standard priors used in the LVK analysis of individual events where $d_L$ is the luminosity distance.

In this work, we incorporate the PBH population distribution \eqref{cR} into the \texttt{ICAROGW} \citep{Mastrogiovanni:2021wsd} package to estimate the likelihood function \eqref{L2}, and use \texttt{dynesty} \citep{Speagle:2019ivv} sampler called from \texttt{Bilby} \citep{Ashton:2018jfp,Romero-Shaw:2020owr} to sample over the parameter space. We use the GW events from GWTC-3 by discarding events with false alarm rate larger than 1 yr$^{-1}$ and events with the secondary component mass smaller than $3\Msun$ to avoid contamination from putative events involving neutron stars following \cite{DeLuca:2021wjr}. A total of $69$ GW events from GWTC-3 meet this criteria and the posterior samples of these BBHs are publicly available from \cite{ligo_scientific_collaboration_and_virgo_2021_5655785}.

%%%%%%%%%%%%%%%%%%%%%%%%%%%%%%%%%%%%%%%%%%%%%%%%%%%%%%%%%%%%%%%%%%%%%%%%%%%%%%%%
\section{\label{result}results}
%%%%%%%%%%%%%%%%%%%%%%%%%%%%%%%%%%%%%%%%%%%%%%%%%%%%%%%%%%%%%%%%%%%%%%%%%%%%%%%%

%%%%%%%%%%%%%%%%%%%%%%%%%%%%%%%%%%%%%%%%%%%%%%%%%%%%%%%%%%%%%%%%%%%%%%%%%%%%%%%%
\begin{table}[tbp!]
	\centering
	\begin{tabular}{lll}
		\hline\hline
		\textbf{Parameter\quad} & \textbf{Description} & \textbf{Prior} \\
		\hline
		$\fpbh$ & Abundance of PBH in dark matter & log-$\mU(-4, 0)$\\
		\hline
		\multicolumn{3}{c}{Lognormal PBH mass function} \\[1pt]
		$\Mc$ & Central mass in $\Msun$. & $\mU(5, 50)$\\
		$\sigma$ & Mass width. & $\mU(0.1, 2)$\\
		\hline
		\multicolumn{3}{c}{Power-law PBH mass function} \\[1pt]
		$\Mmin$ & Lower mass cut-off in $\Msun$. & $\mU(3, 10)$\\
		$\al$ & Power-law index. & $\mU(1.05, 4)$\\
		\hline
		\multicolumn{3}{c}{Broken Power-law PBH mass function} \\[1pt]
		$M_*$ & Peak mass in $\Msun$. & $\mU(3, 15)$\\
		$\al_1$ & First power-law index. & $\mU(0, 15)$\\
		$\al_2$ & Second power-law index. & $\mU(1, 3)$\\
		\hline
		\multicolumn{3}{c}{Critical collapse (CC) PBH mass function} \\[1pt]
		$\Mf$ & Horizon mass scale in $\Msun$. & $\mU(1, 50)$\\
		$\al$ & Universal exponent. & $\mU(0, 5)$\\
		\hline
	\end{tabular}	
	\caption{\label{table:priors}Parameters and their prior distributions used in the Bayesian parameter estimations. Here, $\mU$ and log-$\mU$ denote uniform and log-uniform distributions, respectively.}
\end{table}
Based on the hierarchical statistical framework, we do the parameter estimations for four different PBH mass functions that are commonly used in the literature. These mass functions are the log-normal, power-law, broken power-law, and critical collapse (CC) distributions, respectively. We summarize the parameters and their prior distributions in \Table{table:priors}. Below we show the results for each of the PBH mass function.

\subsection{Log-normal mass function}
We first consider a PBH mass function with the log-normal form \citep{Dolgov:1992pu} of
\e\label{log}
P(m) = \frac{1}{\sqrt{2\pi} \s m} \exp \(-\frac{\ln^2\(m/M_c\)}{2\s^2}\),
\q
where $M_c$ represents the central mass of $m P(m)$, and $\s$ characterizes the width of the mass spectrum.
A huge class of extended mass distributions can be approximated by the log-normal mass function if PBHs are formed from a smooth symmetric peak in the inflationary power spectrum when the slow-roll approximation holds \citep{Green:2016xgy,Carr:2017jsz,Kannike:2017bxn}.
The hyper-parameters are $\Ld = \{M_c, \s, \fpbh\}$ in this case. 
We can then derive the averaged PBH mass and averaged number density from \Eq{mpbh} and \Eq{Fm} as
\e
\mpbh = M_c \exp \(-\frac{\s^2}{2}\),
\q
\e 
F(m) = \frac{M_c}{\sqrt{2\pi} \s m^2} \exp \(-\frac{\s^2}{2}-\frac{\ln^2\(m/M_c\)}{2\s^2}\).
\q

\begin{figure}[tbp!]
	\centering
	\includegraphics[width=0.5\textwidth]{post-log.pdf}
	\caption{\label{posterior-log}The marginalized one- and two-dimensional posterior distributions for hyper-parameters $\{M_c, \s, \fpbh\}$ in the log-normal mass function inferred from GWTC-3. The blue color denotes the results from the first merger only, while the orange denotes the results from both the first and second mergers. The contours represent the $1\sigma$, $2\sigma$ and $3\sigma$ credible regions, respectively.}
\end{figure}

\begin{figure}[htbp!]
	\centering
	\includegraphics[width=0.5\textwidth]{ratio-log.pdf}
	\caption{\label{ratio-log}The ratio of merger rate density from second merger to that from first merger, $\cR_2(t_0, m_1, m_2)/\cR_1(t_0, m_1, m_2)$, as a function of component masses for the log-normal mass function. We have fixed the hyper-parameters $\{M_c, \s, \fpbh\}$ to their best-fit values.}
\end{figure}

Using $69$ BBHs from GWTC-3 and performing the hierarchical Bayesian inference, we obtain $M_c = 17.3^{+2.2}_{-2.0} \Msun$, $\s = 0.71^{+0.10}_{-0.08}$, and $\fpbh = 1.8^{+0.3}_{-0.3} \times 10^{-3}$. In this work, we present result with median value and 90\% equal-tailed credible intervals. The posteriors for the hyper-parameters $\Ld = \{M_c, \s, \fpbh\}$ are shown in \Fig{posterior-log}. Note that we get a larger value of $M_c$ than that inferred from GWTC-1 in \cite{Wu:2020drm} because GWTC-3 contains more heavier BHs than those from GWTC-1. From \Eq{Rt}, we also infer the local merger rate as $R(t_0) = 41^{+16}_{-12} \gpcyr$. The results of local merger rate and abundance of PBHs are consistent with the previous estimations \citep{Chen:2018czv,Chen:2018rzo,Chen:2019irf,Wu:2020drm,Chen:2021nxo,Chen:2022fda}, confirming that CDM cannot be dominated by the stellar-mass PBHs.


In \Fig{ratio-log}, we show the ratio of merger rate density from second merger to the one from first merger, namely $\cR_2(t_0, m_1, m_2)/\cR_1(t_0, m_1, m_2)$, by fixing the hyper-parameters $\{M_c, \s, \fpbh\}$ to their best-fit values. It can be seen that the second merger provides more contribution to the total merger rate density as component mass increases. Even though $\cR_2(t_0, m_1, m_2)/\cR_1(t_0, m_1, m_2)$ can reach as high as $\gtrsim 10\%$, the ratio of merger rate from second merger to that from first merger is $R_2(t_0)/R_1(t_0) = 1.0^{+0.2}_{-0.1}\%$ and is negligible. This is because the major contribution to the merger rate is from the masses less than $50\Msun$, and the correction is negligible in this mass range. Therefore the effect of merger history can be safely ignored when estimating the merger rate of PBH binaries.

%%%%%%%%%%%%%%%%%%%%%%%%%%%%%%%%%%%%%%%%%%%%%%%%%%%%%%%%%%%%%%%%%%%%%%%%%%%%%%%%
\subsection{Power-law mass function}
We next consider a PBH mass function with the power-law form \citep{Carr:1975qj} of
\e\label{power} 
    P(m) = \frac{\al-1}{\Mmin} \(\frac{m}{\Mmin}\)^{-\al},
\q
where $\Mmin$ is the lower-mass cut-off such that $m>\Mmin$, and $\al>1$ is the power-law index. 
The power-law mass function can typically result from a broad or flat power spectrum of the curvature perturbations \citep{DeLuca:2020ioi} during radiation dominated era \citep{Carr:2016drx,Carr:2017jsz}.
The hyper-parameters are $\Ld = \{\Mmin, \al, \fpbh\}$ in this case. 
We can then derive the averaged PBH mass and averaged number density from \Eq{mpbh} and \Eq{Fm} as
\e
    \mpbh = \Mmin \frac{\al}{\al-1},
\q
\e 
    F(m) = \frac{\al}{m} \(\frac{m}{\Mmin}\)^{-\al}.
\q

Using $69$ BBHs from GWTC-3 and performing the hierarchical Bayesian inference, we obtain $\Mmin = 6.5^{+0.3}_{-0.8} \Msun$, $\al = 1.9^{+0.2}_{-0.2}$, and $\fpbh = 2.3^{+0.3}_{-0.3} \times 10^{-3}$. The posteriors for the hyper-parameters $\Ld = \{\Mmin, \al, \fpbh\}$ are shown in \Fig{posterior-power}. Note that we get a smaller value of $\al$ than that inferred from GWTC-1 in \cite{Wu:2020drm} because GWTC-3 contains more heavier BHs than those from GWTC-1. From \Eq{R0}, we also infer the local merger rate as $R_0 = 48^{+15}_{-12} \gpcyr$. The results of local merger rate and abundance of PBHs are consistent with the previous estimations \citep{Chen:2018czv,Chen:2018rzo,Chen:2019irf,Wu:2020drm,Chen:2021nxo,Chen:2022fda}, confirming that CDM cannot be dominated by the stellar-mass PBHs.

\begin{figure}[tbp!]
	\centering
	\includegraphics[width=0.5\textwidth]{post-power.pdf}
	\caption{\label{posterior-power}The marginalized one- and two-dimensional posterior distributions for hyper-parameters $\{\Mmin, \al, \fpbh\}$ in the power-law mass function inferred from GWTC-3. The blue color denotes the results from the first merger only, while the orange denotes the results from both the first and second mergers. The contours represent the $1\sigma$, $2\sigma$ and $3\sigma$ credible regions, respectively.}
\end{figure}

\begin{figure}[htbp!]
	\centering
	\includegraphics[width=0.5\textwidth]{ratio-power.pdf}
	\caption{\label{ratio-power}The ratio of merger rate density from second merger to that from first merger, $\cR_2(t_0, m_1, m_2)/\cR_1(t_0, m_1, m_2)$, as a function of component masses for the power-law mass function. We have fixed the hyper-parameters $\{\Mmin, \al, \fpbh\}$ to their best-fit values.}
\end{figure}

In \Fig{ratio-power}, we show the ratio of merger rate density from second merger to the one from first merger, namely $\cR_2(t_0, m_1, m_2)/\cR_1(t_0, m_1, m_2)$, by fixing the hyper-parameters $\{\Mmin, \al, \fpbh\}$ to their best-fit values. It can be seen that the second merger provides more contribution to the total merger rate density as component mass increases. Even though $\cR_2(t_0, m_1, m_2)/\cR_1(t_0, m_1, m_2)$ can reach as high as $\gtrsim 10\%$, the ratio of merger rate from second merger to that from first merger is $R_2(t_0)/R_1(t_0) = 0.9^{+0.1}_{-0.1}\%$ and is negligible. This is because the major contribution to the merger rate is from the masses less than $50\Msun$, and the correction is negligible in this mass range. Therefore the effect of merger history can be safely ignored when estimating the merger rate of PBH binaries.

%%%%%%%%%%%%%%%%%%%%%%%%%%%%%%%%%%%%%%%%%%%%%%%%%%%%%%%%%%%%%%%%%%%%%%%%%%%%%%%%
\subsection{Broken power-law mass function}
We then consider a PBH mass function with the broken power-law form \citep{Deng:2021ezy} of
\begin{equation}
	P(m)= \left(\frac{m_*}{\alpha_1+1}-\frac{m_*}{\alpha_2-1}\right)^{-1} \begin{cases} (\frac{m}{m_*})^{\alpha_1}, & m<m_* \\ (\frac{m}{m_*})^{-\alpha_2}, & m>m_*\end{cases},
\end{equation}
where $m_*$ is the peak mass of $m P(m)$. Here $\al_1>0$ and $\al_2>1$ are two power-law indices. The broken power-law mass function is a generalization of the power-law form and can be achieved if PBHs are formed by vacuum bubbles that nucleate during inflation via quantum tunneling \citep{Deng:2021ezy}.
The hyper-parameters are $\Ld = \{m_*, \al_1, \al_2, \fpbh\}$ in this case. 
We can then derive the averaged PBH mass and averaged number density from \Eq{mpbh} and \Eq{Fm} as
\e
\mpbh = \frac{\al_1 \al_2}{\(\al_1 +1\)\(\al_2 -1\)} m_*,
\q
\e 
F(m) = \frac{\al_1 \al_2}{\al_1 + \al_2} \begin{cases} (\frac{m}{m_*})^{\alpha_1}, & m<m_* \\ (\frac{m}{m_*})^{-\alpha_2}, & m>m_*\end{cases}.
\q

\begin{figure}[tbp!]
	\centering
	\includegraphics[width=0.5\textwidth]{post-bpower.pdf}
	\caption{\label{posterior-bpower}The marginalized one- and two-dimensional posterior distributions for hyper-parameters $\{m_*, \al_1, \al_2, \fpbh\}$ in the broken power-law mass function inferred from GWTC-3. The blue color denotes the results from the first merger only, while the orange denotes the results from both the first and second mergers. The contours represent the $1\sigma$, $2\sigma$ and $3\sigma$ credible regions, respectively.}
\end{figure}

\begin{figure}[tbp!]
	\centering
	\includegraphics[width=0.5\textwidth]{ratio-bpower.pdf}
	\caption{\label{ratio-bpower}The ratio of merger rate density from second merger to that from first merger, $\cR_2(t_0, m_1, m_2)/\cR_1(t_0, m_1, m_2)$, as a function of component masses for the broken power-law mass function. We have fixed the hyper-parameters $\{m_*, \al_1, \al_2, \fpbh\}$ to their best-fit values.}
\end{figure}

Using $69$ BBHs from GWTC-3 and performing the hierarchical Bayesian inference, we obtain $m_* = 8.6^{+2.1}_{-1.3} \Msun$, $\al_1 = 5.6^{+7.1}_{-3.8}$, $\al_2 = 2.0^{+0.2}_{-0.2}$, and $\fpbh = 2.2^{+0.3}_{-0.4} \times 10^{-3}$. The posteriors for the hyper-parameters $\Ld = \{m_*, \al_1, \al_2, \fpbh\}$ are shown in \Fig{posterior-bpower}. From \Eq{R0}, we also infer the local merger rate as $R_0 = 46^{+15}_{-11} \gpcyr$. The results of local merger rate and abundance of PBHs are consistent with the previous estimations \citep{Chen:2018czv,Chen:2018rzo,Chen:2019irf,Wu:2020drm,Chen:2021nxo,Chen:2022fda}, confirming that CDM cannot be dominated by the stellar-mass PBHs. Note that \cite{Deng:2021ezy} obtains quite different results for the broken power-law mass function using GWTC-2 data without accounting for the uncertainties in the measurement of $\vth$. \cite{Deng:2021ezy} also deals with the selection effect of the GW detectors in an improper way. We emphasize that our results are more robust in the sense that we have a comprehensive treatment of the selection effect and also the measurement uncertainties. Nevertheless, our results of the broken power-law mass function are fully consistent with the results of the power-law case as expected.


In \Fig{ratio-bpower}, we show the ratio of merger rate density from second merger to the one from first merger, namely $\cR_2(t_0, m_1, m_2)/\cR_1(t_0, m_1, m_2)$, by fixing the hyper-parameters $\{m_*, \al_1, \al_2, \fpbh\}$ to their best-fit values. It can be seen that the second merger provides more contribution to the total merger rate density as component mass increases. Even though $\cR_2(t_0, m_1, m_2)/\cR_1(t_0, m_1, m_2)$ can reach as high as $\gtrsim 10\%$, the ratio of merger rate from second merger to that from first merger is $R_2(t_0)/R_1(t_0) = 0.9^{+0.3}_{-0.1}\%$ and is negligible. This is because the major contribution to the merger rate is from the masses less than $50\Msun$, and the correction is negligible in this mass range. Therefore the effect of merger history can be safely ignored when estimating the merger rate of PBH binaries.

%%%%%%%%%%%%%%%%%%%%%%%%%%%%%%%%%%%%%%%%%%%%%%%%%%%%%%%%%%%%%%%%%%%%%%%%%%%%%%%%
\subsection{Critical collapse mass function}
We last consider a PBH mass function with the critical collapse form \citep{Niemeyer:1997mt,Yokoyama:1998xd,Carr:2016hva,Gow:2020cou} of
\e
P(m)=\frac{\alpha^2\,  m^\alpha}{\Mf^{1+\alpha}\, \Gamma(1 / \alpha)} \exp \left(-(m/\Mf)^{\alpha}\right),
\q
where $\alpha$ is a universal exponent relating to the critical collapse of radiation, and $M_{\mathrm{f}}$ is the mass scale at the order of horizon mass at the collapse epoch \citep{Carr:2016hva}. 
For this mass spectrum, there is no lower mass cut-off, but it is exponentially suppressed above the mass scale of $\Mf$.
The critical collapse mass function is closely associated with a $\delta$-function power spectrum of the density fluctuations \citep{Niemeyer:1997mt,Yokoyama:1998xd,Carr:2016hva,Gow:2020cou}. 
The hyper-parameters are $\Ld = \{\Mf, \al, \fpbh\}$ in this case. 
We can then derive the averaged PBH mass and averaged number density from \Eq{mpbh} and \Eq{Fm} as
\e
\mpbh = \frac{\Mf \Gamma(1 / \alpha)}{\alpha},
\q
\e 
F(m) = \al \Mf^{-\al} m^{\al -1} \exp\(-(m/\Mf)^{\al}\).
\q

\begin{figure}[tbp!]
	\centering
	\includegraphics[width=0.5\textwidth]{post-CC.pdf}
	\caption{\label{posterior-CC}The marginalized one- and two-dimensional posterior distributions for hyper-parameters $\{\Mf, \al, \fpbh\}$ in the critical collapse mass function inferred from GWTC-3. The blue color denotes the results from the first merger only, while the orange denotes the results from both the first and second mergers. The contours represent the $1\sigma$, $2\sigma$ and $3\sigma$ credible regions, respectively.}
\end{figure}

\begin{figure}[tbp!]
	\centering
	\includegraphics[width=0.5\textwidth]{ratio-CC.pdf}
	\caption{\label{ratio-CC}The ratio of merger rate density from second merger to that from first merger, $\cR_2(t_0, m_1, m_2)/\cR_1(t_0, m_1, m_2)$, as a function of component masses for the critical collapse mass function. We have fixed the hyper-parameters $\{\Mf, \al, \fpbh\}$ to their best-fit values.}
\end{figure}

Using $69$ BBHs from GWTC-3 and performing the hierarchical Bayesian inference, we obtain $\Mf = 10.8^{+3.7}_{-3.6} \Msun$, $\al = 1.0^{+0.2}_{-0.2}$, and $\fpbh = 1.5^{+0.2}_{-0.2} \times 10^{-3}$. The posteriors for the hyper-parameters $\Ld = \{\Mf, \al, \fpbh\}$ are shown in \Fig{posterior-CC}. From \Eq{R0}, we also infer the local merger rate as $R_0 = 49^{+26}_{-16} \gpcyr$. The results of local merger rate and abundance of PBHs are consistent with the previous estimations \citep{Chen:2018czv,Chen:2018rzo,Chen:2019irf,Wu:2020drm,Chen:2021nxo,Chen:2022fda}, confirming that CDM cannot be dominated by the stellar-mass PBHs. 

In \Fig{ratio-CC}, we show the ratio of merger rate density from second merger to the one from first merger, namely $\cR_2(t_0, m_1, m_2)/\cR_1(t_0, m_1, m_2)$, by fixing the hyper-parameters $\{\Mf, \al, \fpbh\}$ to their best-fit values. It can be seen that the second merger provides more contribution to the total merger rate density as component mass increases. Even though $\cR_2(t_0, m_1, m_2)/\cR_1(t_0, m_1, m_2)$ can reach as high as $\gtrsim 10\%$, the ratio of merger rate from second merger to that from first merger is $R_2(t_0)/R_1(t_0) = 2.2^{+1.3}_{-0.1}\%$ and is negligible. This is because the major contribution to the merger rate is from the masses less than $50\Msun$, and the correction is negligible in this mass range. Therefore the effect of merger history can be safely ignored when estimating the merger rate of PBH binaries.


%%%%%%%%%%%%%%%%%%%%%%%%%%%%%%%%%%%%%%%%%%%%%%%%%%%%%%%%%%%%%%%%%%%%%%%%%%%%%%%%
\section{\label{conclusion}Conclusion}

In this work, we use $69$ BBHs from GWTC-3 to constrain the merger history of PBH binaries by assuming the observed BBHs from LVK are attributed to PBHs. We perform comprehensive Bayesian analyses by considering four commonly used PBH mass functions in literature, namely the log-normal, power-law, broken power-law, and critical collapse mass functions. 

We summarize the key results in \Table{table:BF}. 
It can be seen that the contribution of merger rate from second merger to the total merger rate is less than $5\%$.
Therefore, the merger history of PBH binaries has subdominant effect, and this effect can be neglected  when evaluating the merger rate of PBH binaries.
It can also be seen that the Bayes factors for the model with second merger vesus the model with only the first merger, $\mathrm{BF}^{\mathrm{2nd}}_{\mathrm{1st}}$, are all smaller than $3$, indicating the evidence for the second merger is ``not worth more than a bare mention" \citep{BF}. In this sense, the Bayes factors also imply the effect of merger history can be ignored.

Furthermore, for all the four mass functions, we infer the abundance of PBH in cold dark matter, $\fpbh$, to be at the order of $\mathcal{O}(10^{-3})$. The results of local merger rate and abundance of PBHs are consistent with the previous estimations \citep{Chen:2018czv,Chen:2018rzo,Chen:2019irf,Wu:2020drm,Chen:2021nxo,Chen:2022fda}, confirming that CDM cannot be dominated by the stellar-mass PBHs. 

%%%%%%%%%%%%%%%%%%%%%%%%%%%%%%%%%%%%%%%%%%%%%%%%%%%%%%%%%%%%%%%%%%%%%%%%%%%%%%%%
\begin{table}[tbp!]
	\centering
	\begin{tabular}{l|c|c|c|c}
		\hline\hline
		& LN & PL & BPL & CC \\
		\hline
		$\mathrm{BF}^{\mathrm{2nd}}_{\mathrm{1st}}$ & $0.9$ & $0.4$ & $0.69$ & $1.2$ \\
		$\mathrm{BF}_{\mathrm{PL}}$ & $166$ & $1$ & $2$ & $139$ \\
		$10^{3}\fpbh$ & $1.8^{+0.3}_{-0.3}$ & $2.3^{+0.3}_{-0.3}$ & $2.2^{+0.3}_{-0.4}$ & $1.5^{+0.2}_{-0.2}$ \\
		$10^{2}R_2/R_1$ & $1.0^{+0.2}_{-0.1}$ & $0.9^{+0.1}_{-0.1}$ & $0.9^{+0.3}_{-0.1}$ & $2.2^{+1.3}_{-0.5}$ \\
		\hline
	\end{tabular}
	\caption{\label{table:BF} Summary of the key results for the log-normal (LN), power-law (PL), broken power-law (BPL), and critical collapse (CC) mass functions. The first row, $\mathrm{BF}^{\mathrm{2nd}}_{\mathrm{1st}}$, shows the Bayes factors for the model with 2nd merger versus the model with only 1st merger; the second row, $\mathrm{BF}_{\mathrm{PL}}$, shows the Bayes factors for the model with different PBH mass function versus the model with the power-law PBH mass function by accounting for the second merger effect; the third row, $\fpbh$, shows the abundance of PBH in dark matter inferred from different models by accounting for the second merger effect; the last row, $R_2/R_1$, shows the merger rate ratio between the second merger and the first merger.}
\end{table}
%%%%%%%%%%%%%%%%%%%%%%%%%%%%%%%%%%%%%%%%%%%%%%%%%%%%%%%%%%%%%%%%%%%%%%%%%%%%%%%%

We also compute the Bayes factors between the models with different PBH mass functions. The Bayes factors $\mathrm{BF}_{\mathrm{PL}}$ are estimated by taking the model with power-law mass function as the fiducial model. 
We find that $\mathrm{BF}_{\mathrm{PL}}^{\mathrm{LG}}$ has the largest value, indicating that GWTC-3 can be best fitted by the log-normal mass function among the four mass functions considered in this work. Our findings contradict with the results from \cite{Deng:2021ezy} claiming that broken power-law mass function can fit better than the log-normal form. There are some drawbacks from analyses in \cite{Deng:2021ezy}. Firstly, \cite{Deng:2021ezy} neglects the uncertainties in the measurement of each event's masses and completely ignores redshift evolution of the merger rate. Secondly, \cite{Deng:2021ezy} deals with the selection effect of GW detectors in an improper way. In this sense, we disagree with \cite{Deng:2021ezy} and conclude that the most frequently used log-normal mass function can fit GWTC-3 best among the four mass functions.


%%%%%%%%%%%%%%%%%%%%%%%%%%%%%%%%%%%%%%%%%%%%%%%%%%%%%%%%%%%%%%%%%%%%%%%%%%%%%%%%
%%%%%%%%%%%%%%%%%%%%%%%%%%%%%%% acknowledgments %%%%%%%%%%%%%%%%%%%%%%%%%%%%%%%
%\section*{acknowledgments}
\begin{acknowledgments}
% We thank the referee for providing constructive comments and suggestions to improve the quality of this paper. 
We would also like to thank Xingjiang Zhu, Xiao-Jin Liu, Shen-Shi Du, and Zhu Yi for useful discussions.
Z.-C.C. is supported by the China Postdoctoral Science Foundation Fellowship No.~2022M710429.
This research has made use of data, software and/or web tools obtained from the Gravitational Wave Open Science Center (\url{https://www.gw-openscience.org/}), a service of LIGO Laboratory, the LIGO Scientific Collaboration, the Virgo Collaboration, and the KAGRA Collaboration. 
LIGO Laboratory and Advanced LIGO are funded by the United States National Science Foundation (NSF) as well as the Science and Technology Facilities Council (STFC) of the United Kingdom, the Max-Planck-Society (MPS), and the State of Niedersachsen/Germany for support of the construction of Advanced LIGO and construction and operation of the GEO600 detector. Additional support for Advanced LIGO was provided by the Australian Research Council. Virgo is funded, through the European Gravitational Observatory (EGO), by the French Centre National de Recherche Scientifique (CNRS), the Italian Istituto Nazionale di Fisica Nucleare (INFN) and the Dutch Nikhef, with contributions by institutions from Belgium, Germany, Greece, Hungary, Ireland, Japan, Monaco, Poland, Portugal, Spain. The construction and operation of KAGRA are funded by Ministry of Education, Culture, Sports, Science and Technology (MEXT), and Japan Society for the Promotion of Science (JSPS), National Research Foundation (NRF) and Ministry of Science and ICT (MSIT) in Korea, Academia Sinica (AS) and the Ministry of Science and Technology (MoST) in Taiwan.
\end{acknowledgments}	
	
%%%%%%%%%%%%%%%%%%%%%%%%%%%%%%%%%%%%%%%%%%%%%%%%%%%%%%%%%%%%%%%%%%%%%%%%%%%%%%%%
%%%%%%%%%%%%%%%%%%%%%%%%%%%%%%% %%% references %%%%%%%%%%%%%%%%%%%%%%%%%%%%%%%%%%
\bibliography{ref}

\end{document}
%%%%%%%%%%%%%%%%%%%%%%%%%%%%%%%%%%%%%%%%%%%%%%%%%%%%%%%%%%%%%%%%%%%%%%%%%%%%%%%%