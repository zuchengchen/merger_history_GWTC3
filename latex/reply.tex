%%%%%%%%%%%%%%%%%%%%%%%%%%%%%%%%%%%%%%%%%%%%%%%%%%%%%%%%%%%%%%%%%%%%%%%%%%%%%%%%
\documentclass[
%a4paper,       % letterpaper, a4paper, a5paper
%10pt,              % 10pt, 11pt, 12pt
%reprint,           % journal's actual layout
preprint,          % 12pt, single-column
superscriptaddress,% authors with affiliations via superscripts
amsmath,           % add AMS-Latex features
amssymb,           % add extra AMS symbols, including amsfonts
aps,               % aps or aip
prl,               % prl, pra, prb, prc, prd, pre, prstab
%showpacs,          % make PACS codes appear
notitlepage,       % control appearance of title page
longbibliography,  % show  article titles in the bibliography
floatfix,          % process floats as early as possible
%showkeys,          % option to make keywords appear
%titlepage,         %
%eqsecnum,          % number equations by section
nofootinbib,
onecolumn,
]{revtex4-1}

\usepackage{amsmath}
\usepackage{graphicx}   % include figures
\usepackage[
colorlinks=true,        % color link
citecolor=blue,         % cite color
linkcolor=blue,         % link color
urlcolor=blue           % url color
]{hyperref}             % create hyperlinks

\begin{document}

Dear Editor and Referee, \\

We thank the referee for careful reading of our manuscript and for his/her insightful suggestions and comments. 
We enclose our reply to comments in the following part of our reply. 
\begin{enumerate}
	\item The power-law and log-normal mass functions are commonly used in literature and are worthwhile to investigate because they are well motivated by the formation mechanisms of primordial black holes (PBHs). The mass function of PBHs is closely related to shape of primordial power spectrum when PBHs are formed. It is believed that power-law and log-normal mass functions can be representatives of quite a large class of mass distributions. We added corresponding explanations after Eq.(23) and Eq.(26), respectively, to address the motivations of using these two mass functions.
    
    \item The text after Eq.~(4) to the end of Section 2 was completely rewritten as suggested by the referee. We added some necessary intermediate steps and explanations to make the derivations much more readable.
    
    \item We added units to the labels of figures and enlarge labels' font size. Furthermore, more explanations were added to the captions.
    
    \item We corrected the grammatical errors as pointed out by the referee.
    
\end{enumerate}



Please reconsider our paper for publication in PRD. \\


Best wishes, 

You Wu 

\newpage

%%%%%%%%%%%%%%%%%%%%%%%%%%%%%%%%%%%%%%%%
%%%%%%%%%%%%%%%%%%%%%%%%%%%%%%%%%%%%%%%%
\end{document}